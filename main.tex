\documentclass[pdf]{beamer}

\usepackage{helvet}
\usepackage[english]{babel}
\usepackage{pgfplots}
\usepackage{pgf}
\usepackage{tikz}
\pgfplotsset{compat=newest}
\usepackage{booktabs}
\usepackage{graphics}
\usepackage[T1]{fontenc}
\usepackage[utf8]{inputenc}
\usepackage{lipsum}
\usepackage{tcolorbox}
\usepackage{xcolor}
\usepackage{listings}
\usepackage{microtype}
\usepackage{float}
\usepackage{siunitx}
\usepackage{multicol}
\usepackage{hyperref}
\usepackage{dsfont}
\usepackage{caption}
\usepackage{subcaption}
\usepackage{graphicx}
\usepackage{amsmath}
\usepackage[sorting=nty]{biblatex}
\usepackage{hyperref}

% usetheme first to capture the \firstcolor and \secondcolor
\usetheme{ntu}

% implement `setcolor` function to quickly set theme colors
\newcommand{\setcolor}[2][]{%
    % #1 : optional, the secondary color [ntugold]
    % #2 : mandatory, the main color theme [ntubrick]
    % If the first argument is blank, it will be a mono-color theme
    \ifx\relax#1\relax
        \def\firstcolor{#2} \def\secondcolor{#2}
    \else
        \def\firstcolor{#2} \def\secondcolor{#1}
    \fi
}

% implement `setdarkframecolor` to quickly set darkframe background color
\newcommand{\setdarkframecolor}[1]{\def\darkframecolor{#1}}
\def\darkframecolor{darkthemeblack}


% implement new cite function that can manually set the level of importance
% importance level including {A, S}
% if the first arg is missing, it will behave exactly like normal \cite{}
\let\oldcite\cite
\renewcommand{\cite}[2][]{%
    % #1 : otional, citation importance level (A, S)
    % #2 : citation doc
    \ifx\relax#1\relax
        \oldcite{#2}
    \else
        \ifthenelse{\equal{#1}{S}}{{\textbf{\color{\firstcolor} \cite{#2}}}}{{\color{\firstcolor} \cite{#2}}}    
    \fi
}

% A customized `examplebox` automatically set to secondary color
\newcommand{\examplebox}[2]{%
% here use tcolorbox package's tcolorbox environment
\begin{tcolorbox}[colframe=\secondcolor,colback=\secondcolor!10,title=#1]
#2
\end{tcolorbox}}


% A more flexible box environment called mybox
% To use it :
% \begin{mybox}{color}{title}
%   content
% \end{mybox}
\newtcolorbox{mybox}[3][]{
    % #1 : content
    % #2 : color
    % #3 : title
    colframe = #2!25,
    colback  = #2!10,
    coltitle = #2!20!black,
    title    = #3,
    #1,
}

% implement the \darkthemelogo function to plot white logo in dark bg
\newcommand{\darkthemelogo}{
 \begin{tikzpicture}[remember picture, overlay]
        \node[anchor=north west, 
              xshift=0.6mm, 
              yshift=-3.0mm] 
             at (current page.north west) 
             {\includegraphics[width=9mm,keepaspectratio]{logos/dark_theme_logo.png}}; 
        \end{tikzpicture}
}





% keep reference template \setdepartment command
\newcommand{\setdepartment}[1]{\def\department{#1}}

\setbeamersize{text margin left=22mm}
\def\insertframetitle{}

% keep reference template command to plot title page

\newcommand{\inserttitlepage}{
% big white logo on the first page
% If you want to remove, plz comment the whole frame
    \begin{frame}[plain, noframenumbering]{}

        \begin{center}
       \hspace{-3.6em}
         \includegraphics[width = 0.6\paperwidth]{logos/\targetcolourmodel/ntu_logo.png}
       \end{center}
    \end{frame}

    \begin{frame}[plain]{}
        \color{white}\maketitle    
    \end{frame}

    \setbeamercolor{background canvas}{bg = white}
}


% Darkframe Setups
\newenvironment{darkframe}[1][]
    {
    % \setcolor[darkthemepurple]{darkthemepink}
    \setbeamercolor{background canvas}{bg=\darkframecolor} % Bg
    \setbeamercolor{hl}{\secondcolor!20!ntudark}
    \setbeamercolor{section}{fg=\secondcolor!70!ntudark}
    \setbeamercolor{frametitle}{fg = darkthemewhite}
    \setbeamercolor{enumerate}{fg = darkthemewhite}
    \setbeamercolor{itemize}{bg = darkthemewhite}
    \setbeamercolor{itemize/enumerate body}{fg = darkthemewhite}
    \setbeamercolor{caption}{fg=darkthemewhite}
    \begin{frame}{#1}
    \color{darkthemewhite}
    \darkthemelogo
    }
    { 
    \end{frame}
    }
    
% \hypersetup{colorlinks,linkcolor=\secondcolor, urlcolor=\secondcolor, citecolor=\secondcolor}



% Code displaying setup through listings

% basic setups
\lstset{%
    basicstyle=\footnotesize\ttfamily,
    keywordstyle=\color{\secondcolor}\bfseries\ttfamily,
    backgroundcolor = \color{lightgray!20},
    stringstyle = \color{\secondcolor},
    commentstyle=\color{\firstcolor!60},
    numbers=left,
    xleftmargin=1.0ex,
    stepnumber=1,
    numbersep=10pt,
    tabsize=4,
    showspaces=false,
    showstringspaces=false
}

% Python style
\lstdefinestyle{py}{
  language=python,
  numbers=left,
  xleftmargin=1.0ex,
  stepnumber=1,
  numbersep=10pt,
  tabsize=4,
  showspaces=false,
  showstringspaces=false
}

% R style
\lstdefinestyle{R}{
  language=R,
  numbers=left,
  xleftmargin=1.0ex,
  stepnumber=1,
  numbersep=10pt,
  tabsize=4,
  showspaces=false,
  showstringspaces=false
}

% TeX style
\lstdefinestyle{tex}{
  language=TeX,
  xleftmargin=1.0ex,
  stepnumber=1,
  numbersep=10pt,
  tabsize=4,
  showspaces=false,
  showstringspaces=false
}

% NOTE : In fact, directly using lstlisting would be quick enough
% FIXME : failed code environment, use lstlisting would be faster
% \newenvironment{mycode}[1][\unskip]
%     {
%     \begin{lstlisting}[style=#1]
%     }
%     { 
%     \end{lstlisting}
%     }

% NOTE: self-defined resized table
% \newenvironment{mytable}[1][1]
%     {
%     \begin{table}
%     \resizebox{#1\columnwidth}{!}{
%     \begin{tabular}{c|c}
%          &  \\
%          & 
%     \end{tabular}
%     }
%     { 
%     \end{table}
%     }


% The end of preamle file
\usepackage{fontawesome}
\usepackage{FiraSans}
\usepackage{FiraMono}
%TODO: 改 thm and box color to second


% \setbeamercovered{transparent}


% \usetheme{Madrid}
\usecolortheme{ntu}

\usetheme{ntu}
% \setcolor[ntudark]{ntured}
\setcolor[ntugold]{ntubrick}
% \setcolor[blue]{deepblue}
% \setcolor[ntured]{ntudark}



\hypersetup{colorlinks,linkcolor=\secondcolor, urlcolor=\secondcolor, citecolor=\secondcolor}

\title{NTU \\BEAMER TEMPLATE}
\subtitle{A NTU beamer template}
\author{Liang-Cheng Chen \& Po-Kang Hsiao}
\date{June 2021}



% \AtBeginSection[]{
%   \begin{frame}
%   \vfill
%   \centering
%   \begin{beamercolorbox}[sep=8pt,center]{title}
%     \usebeamerfont{title}\insertsectionhead\par%
%   \end{beamercolorbox}
%   \vfill
%   \end{frame}
% }

% \AtBeginSection{%
%     \begin{frame}[plain, noframenumbering]{}
%         \insertsectionhead
%     \end{frame}%
    
% }

\bibliography{reference.bib}
\renewcommand*{\bibfont}{\footnotesize}

\hypersetup{colorlinks,linkcolor=,urlcolor=\firstcolor}

\begin{document}

\inserttitlepage

\begin{frame}{Outline}

    \tableofcontents
    
\end{frame}


\section{Introduction}

\begin{frame}{About\, this\, project ...}

    \begin{itemize}
        \item A less beamer-like NTU beamer template 
        \item Easy color switching
        \item Literature importance marking
        \item Colorized description box
    \end{itemize}
    
\end{frame}


\section{Features}

\subsection{Colors}

\begin{frame}{Colors}

\begin{itemize}
    \item Use pre-defined NTU colors
    \item Add new colors to \alert{beamercolorthementu.sty} file
\end{itemize}
\vspace{.5cm}
\begin{mybox}{gray}{This template provides below colors to choose}
    \begin{testcolors}[rgb]
        
        \testcolor{ntubrick}
        \testcolor{ntugold}
        
        \testcolor{ntudark}
        \testcolor{ntured}
        \testcolor{ntublue}
        
    \end{testcolors}
    
    
    
\end{mybox} 
\scriptsize{*More combination of colors can be found here : \href{https://colorhunt.co}{Color Hunt}}
\end{frame}



\begin{frame}[fragile]{Change Color}

\begin{itemize}

\item Use \alert{setcolor} function to change the main and secondary color of the slides
    
\begin{lstlisting}[style=tex]
% Use NTU theme
\usetheme{ntu}

% Set 2 colors
\setcolor[SecondColor]{MainColor}

% Set main color only
\setcolor{MainColor}
\end{lstlisting}

\item The colors in top bar and footline would be slightly darker
    
\end{itemize}

\end{frame}

\subsection{Itemize/Enumerate}

\begin{frame}{Item and Enumerate}

    \begin{itemize}
        \item The item color 
        \item The subitem would be closer to secondary color
        \begin{itemize}
            \item subitem color 
        \end{itemize}
        
    \end{itemize}
    
    \begin{enumerate}
        \item Same color is set to enumerate as well
        \item The subitem will be closer to secondary
        \begin{enumerate}
            \item subitem color here
        \end{enumerate}
        
    \end{enumerate}
    
\end{frame}

\subsection{Box}

\begin{frame}{Box}

\examplebox{Example box}{
    This \alert{examplebox} command would take 2 arguments:
    \begin{itemize}
        \item box title
        \item box content
    \end{itemize}
    And the box will automatically set to secondary color.
}

\end{frame}

\begin{frame}{Box}
    
\begin{mybox}{\firstcolor}{More flexible box}
Use mybox enviroment for more colors.
\begin{itemize}
    \item Set to firstcolor
    \item Set to arbitary colors
\end{itemize}
\end{mybox}

\begin{mybox}{deepblue}{Like This box}
mybox\{color\}\{title\}
\end{mybox}
    
\end{frame}

\begin{frame}{More on Box}

\begin{columns}[t]
    
    \begin{column}{0.5\linewidth}
    
        \examplebox{box 1}{This is the box 1.
        \begin{itemize}
            \item First
            \item Second
            \item Third
        \end{itemize}
        
        }
        
        
    \end{column}
    
    \begin{column}{0.33\linewidth}
    
        \examplebox{box 2}{This is the box 2.}
        
    \end{column}

\end{columns}

    
\end{frame}


\begin{frame}{Theorem}
    \begin{Theorem} %TODO: insert thm number
    
        $a^2 + b^2 = c^2$
    
    \end{Theorem}
\end{frame}


\begin{frame}{Block}
    
    \begin{block}{Random Block Title}
    This will be a block \cite{10.1257/aer.20200451}
    This will be another citation \cite{10.1257/aer.20190553}
    This will also be another citation \cite{Adler:1998:MDL}
    \end{block}
    
\end{frame}

\subsection{Overlay}

\begin{frame}{Overlay}

    \begin{itemize}[<+->]
        \item First thing first
        \item Second point 
        \item And if you want to highlight current object
    \end{itemize}
    
    \begin{enumerate}[<+-|alert@+>]
        \item Adding [<|alert@+>] could help coloring
        \item Now we're talking about this point
        \item Test a math expression
        
        \begin{equation*}
            \beta = (X'X)^{-1} \cdot (X'Y)
        \end{equation*}
        
    \end{enumerate}
    
    
\end{frame}

\subsection{Code}

\begin{frame}[fragile]{Code Display}
    
%TODO : code enviroment
    
\begin{lstlisting}[basicstyle=\normalise, style=py]
# Python Code
import torch
import torch.nn as nn
# model architecture
class Classifier(nn.Module):
    def __init__(self):
        super().__init__()
        pass
model = Classifier()
\end{lstlisting}
    
\end{frame}

\subsection{Related Literature}

\begin{frame}{Test Ref Command}



% \let\oldcite\cite
% \renewcommand{\cite}[2][]{%
%     % #1 : otional, citation importance level (a, s)
%     % #2 : citation doc
%     \ifx\relax#1\relax
%         \oldcite{#2}
%     \else
%         \ifthenelse{\equal{#1}{s}}{{\textbf{\color{\firstcolor} \cite{#2}}}}{{\color{\firstcolor} \cite{#2}}}    
%     \fi
% }

% This is a test \textbf{\color{\firstcolor} \cite{10.1257/aer.20181607}}\\
% Another citing test ~{\color{\firstcolor} \cite{10.1257/aer.20190553}}\\
% I will cite this third thing ~\cite{10.1257/aer.20200451}


This is the most important citation ~\cite[s]{10.1257/aer.20181607}\\
This would be second ~\cite[a]{10.1257/aer.20190553}\\
Normal cite ~\cite{Adler:1998:MDL}


\end{frame}




\begin{frame}{Related Literature}
    
\end{frame}


\begin{frame}[t, allowframebreaks]{Reference}

\printbibliography
    
    
\end{frame}

\section{Project}

\begin{frame}{Download}

    To use this template, you could download the whole project from
    our github page.
    % \url{https://github.com/liangchenn/NTU-Beamer}\\
    \hyperlink{https://github.com/liangchenn/NTU-Beamer}{\beamergotobutton{github link}}%TODO: change github link
    \\
    \hyperlink{https://github.com/liangchenn/NTU-Beamer}{\faGithub}
    
\end{frame}



\end{document}
