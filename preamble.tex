\usepackage{helvet}
\usepackage[english]{babel}
\usepackage{pgfplots}
\usepackage{pgf}
\pgfplotsset{compat=newest}
\usepackage{booktabs}
\usepackage{graphics}
\usepackage[T1]{fontenc}
\usepackage[utf8]{inputenc}
\usepackage{lipsum}
\usepackage{tcolorbox}
\usepackage{xcolor}
\usepackage{listings}
\usepackage{microtype}
\usepackage{float}
\usepackage{siunitx}
\usepackage{multicol}
\usepackage{hyperref}
\usepackage{dsfont}
\usepackage{caption}
\usepackage{subcaption}
\usepackage{graphicx}
\usepackage{amsmath}
\usepackage{biblatex}
\usepackage{hyperref}

\usetheme{ntu}
% \newcommand{\setcolor}[2]{\def\chosencolor{#1}, \def\chosencolortwo{#2}}

% \newcommand{\setcolor}[1]{\def\chosencolor{#1}}

% \newcommand{\setcolor}[2]{\def\firstcolor{#1} \def\secondcolor{#2}}
% \hypersetup{colorlinks,linkcolor=\firstcolor, urlcolor=\firstcolor}

\newcommand{\setcolor}[2][]{
    \ifx\relax#1\relax
        \def\firstcolor{#2} \def\secondcolor{#2}
    \else
        \def\firstcolor{#2} \def\secondcolor{#1}
    \fi
}

\let\oldcite\cite
\renewcommand{\cite}[2][]{%
    % #1 : otional, citation importance level (a, s)
    % #2 : citation doc
    \ifx\relax#1\relax
        \oldcite{#2}
    \else
        \ifthenelse{\equal{#1}{s}}{{\textbf{\color{\firstcolor} \cite{#2}}}}{{\color{\firstcolor} \cite{#2}}}    
    \fi
}


% \newcommand{\setbg}[1]{\def\chosenbgcolor{#1}}
%\newcommand{\setcolortwo}[1]{\def\chosencolortwo{#1}}

\newcommand{\setdepartment}[1]{\def\department{#1}}

% \usetheme{ntu}
\setbeamersize{text margin left=22mm}
\def\insertframetitle{}

\newcommand{\inserttitlepage}{
    \begin{frame}[plain, noframenumbering]{}

        \begin{center}
        \hspace{-3.6em}
         \includegraphics[width = 0.6\paperwidth]{logos/\targetcolourmodel/ntu_logo.png}
%            \includegraphics[width = 0.25\paperwidth]{logos/\targetcolourmodel/white.pdf}
        \end{center}
        
    \end{frame}

    \begin{frame}[plain]{}
        \color{white}\maketitle    
    \end{frame}

    \setbeamercolor{background canvas}{bg = white}
}


\newcommand{\examplebox}[2]{
\begin{tcolorbox}[colframe=\secondcolor,colback=\secondcolor!10,title=#1]
#2
\end{tcolorbox}}


\newtcolorbox{mybox}[3][]{
    colframe = #2!25,
    colback  = #2!10,
    coltitle = #2!20!black,
    title    = #3,
    #1,
}

% \newtcolorbox{example}[3][]{
%     colframe = #2!25,
%     colback  = #2!10,
%     coltitle = #2!20!black,
%     title    = #3,
%     #1,
% }


\lstset{%
    basicstyle=\normalise\ttfamily,
    keywordstyle=\color{\secondcolor}\bfseries\ttfamily,
    backgroundcolor = \color{lightgray!20},
    stringstyle = \color{\secondcolor},
    commentstyle=\color{\firstcolor!60}
}

\lstdefinestyle{py}{
  language=python,
  numbers=left,
  xleftmargin=1.0ex,
  stepnumber=1,
  numbersep=10pt,
  tabsize=4,
  showspaces=false,
  showstringspaces=false
}

\lstdefinestyle{R}{
  language=R,
  numbers=left,
  xleftmargin=1.0ex,
  stepnumber=1,
  numbersep=10pt,
  tabsize=4,
  showspaces=false,
  showstringspaces=false
}


\lstdefinestyle{tex}{
  language=TeX,
  xleftmargin=1.0ex,
  stepnumber=1,
  numbersep=10pt,
  tabsize=4,
  showspaces=false,
  showstringspaces=false
}



% \lstset{%
%     language=Latex,
%     basicstyle=\ttfamily,
%     keywordstyle=\color{\secondcolor}\bfseries\ttfamily,
%     breaklines=true
% }

