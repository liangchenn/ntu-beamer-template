\usepackage{helvet}
\usepackage[english]{babel}
\usepackage{pgfplots}
\usepackage{pgf}
\usepackage{tikz}
\pgfplotsset{compat=newest}
\usepackage{booktabs}
\usepackage{graphics}
\usepackage[T1]{fontenc}
\usepackage[utf8]{inputenc}
\usepackage{lipsum}
\usepackage{tcolorbox}
\usepackage{xcolor}
\usepackage{listings}
\usepackage{microtype}
\usepackage{float}
\usepackage{siunitx}
\usepackage{multicol}
\usepackage{hyperref}
\usepackage{dsfont}
\usepackage{caption}
\usepackage{subcaption}
\usepackage{graphicx}
\usepackage{amsmath}
\usepackage[sorting=nty]{biblatex}
\usepackage{hyperref}

% usetheme first to capture the \firstcolor and \secondcolor
\usetheme{ntu}

% implement `setcolor` function to quickly set theme colors
\newcommand{\setcolor}[2][]{%
    % #1 : optional, the secondary color [ntugold]
    % #2 : mandatory, the main color theme [ntubrick]
    % If the first argument is blank, it will be a mono-color theme
    \ifx\relax#1\relax
        \def\firstcolor{#2} \def\secondcolor{#2}
    \else
        \def\firstcolor{#2} \def\secondcolor{#1}
    \fi
}

% implement `setdarkframecolor` to quickly set darkframe background color
\newcommand{\setdarkframecolor}[1]{\def\darkframecolor{#1}}
\def\darkframecolor{darkthemeblack}


% implement new cite function that can manually set the level of importance
% importance level including {A, S}
% if the first arg is missing, it will behave exactly like normal \cite{}
\let\oldcite\cite
\renewcommand{\cite}[2][]{%
    % #1 : otional, citation importance level (A, S)
    % #2 : citation doc
    \ifx\relax#1\relax
        \oldcite{#2}
    \else
        \ifthenelse{\equal{#1}{S}}{{\textbf{\color{\firstcolor} \cite{#2}}}}{{\color{\firstcolor} \cite{#2}}}    
    \fi
}

% A customized `examplebox` automatically set to secondary color
\newcommand{\examplebox}[2]{%
% here use tcolorbox package's tcolorbox environment
\begin{tcolorbox}[colframe=\secondcolor,colback=\secondcolor!10,title=#1]
#2
\end{tcolorbox}}


% A more flexible box environment called mybox
% To use it :
% \begin{mybox}{color}{title}
%   content
% \end{mybox}
\newtcolorbox{mybox}[3][]{
    % #1 : content
    % #2 : color
    % #3 : title
    colframe = #2!25,
    colback  = #2!10,
    coltitle = #2!20!black,
    title    = #3,
    #1,
}

% implement the \darkthemelogo function to plot white logo in dark bg
\newcommand{\darkthemelogo}{
 \begin{tikzpicture}[remember picture, overlay]
        \node[anchor=north west, 
              xshift=0.6mm, 
              yshift=-3.0mm] 
             at (current page.north west) 
             {\includegraphics[width=9mm,keepaspectratio]{logos/dark_theme_logo.png}}; 
        \end{tikzpicture}
}





% keep reference template \setdepartment command
\newcommand{\setdepartment}[1]{\def\department{#1}}

\setbeamersize{text margin left=22mm}
\def\insertframetitle{}

% keep reference template command to plot title page

\newcommand{\inserttitlepage}{
% big white logo on the first page
% If you want to remove, plz comment the whole frame
    \begin{frame}[plain, noframenumbering]{}

        \begin{center}
       \hspace{-3.6em}
         \includegraphics[width = 0.6\paperwidth]{logos/\targetcolourmodel/ntu_logo.png}
       \end{center}
    \end{frame}

    \begin{frame}[plain]{}
        \color{white}\maketitle    
    \end{frame}

    \setbeamercolor{background canvas}{bg = white}
}


% Darkframe Setups
\newenvironment{darkframe}[1][]
    {
    % \setcolor[darkthemepurple]{darkthemepink}
    \setbeamercolor{background canvas}{bg=\darkframecolor} % Bg
    \setbeamercolor{hl}{\secondcolor!20!ntudark}
    \setbeamercolor{section}{fg=\secondcolor!70!ntudark}
    \setbeamercolor{frametitle}{fg = darkthemewhite}
    \setbeamercolor{enumerate}{fg = darkthemewhite}
    \setbeamercolor{itemize}{bg = darkthemewhite}
    \setbeamercolor{itemize/enumerate body}{fg = darkthemewhite}
    \setbeamercolor{caption}{fg=darkthemewhite}
    \begin{frame}{#1}
    \color{darkthemewhite}
    \darkthemelogo
    }
    { 
    \end{frame}
    }
    
% \hypersetup{colorlinks,linkcolor=\secondcolor, urlcolor=\secondcolor, citecolor=\secondcolor}



% Code displaying setup through listings

% basic setups
\lstset{%
    basicstyle=\footnotesize\ttfamily,
    keywordstyle=\color{\secondcolor}\bfseries\ttfamily,
    backgroundcolor = \color{lightgray!20},
    stringstyle = \color{\secondcolor},
    commentstyle=\color{\firstcolor!60},
    numbers=left,
    xleftmargin=1.0ex,
    stepnumber=1,
    numbersep=10pt,
    tabsize=4,
    showspaces=false,
    showstringspaces=false
}

% Python style
\lstdefinestyle{py}{
  language=python,
  numbers=left,
  xleftmargin=1.0ex,
  stepnumber=1,
  numbersep=10pt,
  tabsize=4,
  showspaces=false,
  showstringspaces=false
}

% R style
\lstdefinestyle{R}{
  language=R,
  numbers=left,
  xleftmargin=1.0ex,
  stepnumber=1,
  numbersep=10pt,
  tabsize=4,
  showspaces=false,
  showstringspaces=false
}

% TeX style
\lstdefinestyle{tex}{
  language=TeX,
  xleftmargin=1.0ex,
  stepnumber=1,
  numbersep=10pt,
  tabsize=4,
  showspaces=false,
  showstringspaces=false
}

% NOTE : In fact, directly using lstlisting would be quick enough
% FIXME : failed code environment, use lstlisting would be faster
% \newenvironment{mycode}[1][\unskip]
%     {
%     \begin{lstlisting}[style=#1]
%     }
%     { 
%     \end{lstlisting}
%     }

% NOTE: self-defined resized table
% \newenvironment{mytable}[1][1]
%     {
%     \begin{table}
%     \resizebox{#1\columnwidth}{!}{
%     \begin{tabular}{c|c}
%          &  \\
%          & 
%     \end{tabular}
%     }
%     { 
%     \end{table}
%     }


% The end of preamle file